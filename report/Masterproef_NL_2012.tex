% ---------- Titelblad Masterproef Faculteit Wetenschappen -----------
% Dit document is opgesteld voor compilatie met pdflatex.  Indien je
% wilt compileren met latex naar dvi/ps, dien je de figuren naar
% (e)ps-formaat om te zetten.
%                           -- december 2012
% -------------------------------------------------------------------
\RequirePackage{fix-cm}
\documentclass[12pt,a4paper,oneside]{book}

% --------------------- In te laden pakketten -----------------------
% Deze kan je eventueel toevoegen aan de pakketten die je al inlaadt
% als je dit titelblad integreert met de rest van thesis.
% -------------------------------------------------------------------
\usepackage{graphicx,xcolor,textpos}
\usepackage{longtable}
\usepackage{helvet}
\usepackage{glossaries}

% -------------------- Pagina-instellingen --------------------------
% Indien je deze wijzigt, zal het titelblad ook wijzigen.  Dit dien je
% dan manueel aan te passen.
% --------------------------------------------------------------------

\topmargin -10mm
\textwidth 160truemm
\textheight 240truemm
\oddsidemargin 0mm
\evensidemargin 0mm

% ------------------- textpos-instellingen ---------------------------
% Enkele andere instellingen voor het voorblad.
% --------------------------------------------------------------------

\definecolor{green}{RGB}{172,196,0}
\definecolor{bluetitle}{RGB}{29,141,176}
\definecolor{blueaff}{RGB}{0,0,128}
\definecolor{blueline}{RGB}{82,189,236}
\setlength{\TPHorizModule}{1mm}
\setlength{\TPVertModule}{1mm}

% ------------------- glossary-instellingen ---------------------------
% abbreviations and Nomenclature
% --------------------------------------------------------------------

% for the nomenclature (comment out if you do not use the nomencl package
\usepackage{nomencl}   % For nomenclature
\renewcommand{\nomname}{List of Symbols}
\newcommand{\myprintnomenclature}{%
  \cleardoublepage%
  \printnomenclature%
  \chaptermark{\nomname}
  \addcontentsline{toc}{chapter}{\nomname} %% comment to exclude from TOC
}
\makenomenclature%

% for the list of abbreviations (comment out if you do not use the glossaries package
\usepackage{glossaries} % For list of abbreviations
\newcommand{\glossname}{List of Abbreviations}
\newcommand{\myprintglossary}{%
  \renewcommand{\glossaryname}{\glossname}
  \cleardoublepage%
  \printglossary[title=\glossname]
  \chaptermark{\glossname}
  \addcontentsline{toc}{chapter}{\glossname} %% comment to exclude from TOC
}
\makeglossaries%

\begin{document}

% ---------------------- Voorblad ------------------------------------
% Vergeet niet de tekst aan te passen:
% - Titel en, indien van toepassing, ondertitel
%          voor eventuele formules in de titel of ondertitel
%          gebruik je  \form{$...$}
% - Je naam
% - Je (co)promotor, begeleider (indien van toepassing)
% - Je opleiding
% - Het academiejaar
% --------------------------------------------------------------------
\thispagestyle{empty}
\newcommand{\form}[1]{\scalebox{1.087}{\boldmath{#1}}}
\sffamily
%
\begin{textblock}{191}(-24,-11)
\colorbox{green}{\hspace{123mm}\ \parbox[c][18truemm]{68mm}{\textcolor{white}{FACULTEIT WETENSCHAPPEN}}}
\end{textblock}
%
\begin{textblock}{70}(-18,-19)
\textblockcolour{}
\includegraphics*[height=19.8truemm]{LogoKULeuven}
\end{textblock}
%
\begin{textblock}{160}(-6,63)
\textblockcolour{}
\vspace{-\parskip}
\flushleft
\fontsize{40}{42}\selectfont \textcolor{bluetitle}{Titel \form{$a^2+b^2=c^2$}}\\[1.5mm]
\fontsize{20}{22}\selectfont Ondertitel \form{$S=\pi r^2$\textsl{(facultatief)}}
\end{textblock}
%
\begin{textblock}{79}(50,103)
\textblockcolour{}
\vspace{-\parskip}
\flushleft
\fbox{\parbox{79mm}{De achtergrond kan wit blijven of je kan een afbeelding invoegen (maximum hoogte 10 cm, breedte variabel, denk aan auteursrechten\ldots). GEEN logo's (je kan binnenin de masterproef logo's gebruiken, maar niet op de voor- of achterpagina). \textit{Verwijder deze tekstkader.}}}
\end{textblock}
%
\begin{textblock}{160}(8,153)
\textblockcolour{}
\vspace{-\parskip}
\flushright
\fontsize{14}{16}\selectfont \textbf{Voornaam ACHTERNAAM}
\end{textblock}
%
\begin{textblock}{70}(-6,191)
\textblockcolour{}
\vspace{-\parskip}
\flushleft
Promotor: Prof. A. Xyz\\[-2pt]
\textcolor{blueaff}{Affiliatie \textsl{(facultatief)}}\\[5pt]
Co-promotor: \textsl{(facultatief)}\\[-2pt]
\textcolor{blueaff}{Affiliatie \textsl{(facultatief)}}\\[5pt]
Begeleider: \textsl{(facultatief)}\\[-2pt]
\textcolor{blueaff}{Affiliatie \textsl{(facultatief)}}\\
\end{textblock}
%
\begin{textblock}{160}(8,191)
\textblockcolour{}
\vspace{-\parskip}
\flushright
Proefschrift ingediend tot het\\[4.5pt]
behalen van de graad van\\[4.5pt]
Master of Science in Xxx\\
\end{textblock}
%
\begin{textblock}{160}(8,232)
\textblockcolour{}
\vspace{-\parskip}
\flushright
Academiejaar 20XX-20XX
\end{textblock}
%
\begin{textblock}{191}(-24,248)
{\color{blueline}\rule{550pt}{5.5pt}}
\end{textblock}
%
TODO
Uitgegeven in eigen beheer, TODO (Belgium)

Alle rechten voorbehouden. Niets uit deze uitgave mag worden vermenigvuldigd en/of openbaar gemaakt
worden door middel van druk, fotokopie, microfilm, elektronisch
of op welke andere wijze ook zonder voorafgaande schriftelijke
toestemming van de uitgever.
All rights reserved. No part of the publication may be reproduced
in any form by print, photoprint, microfilm, electronic or any other means
without written permission from the publisher.

\vfill

\newpage

% Als je het titelblad wil integreren met de rest van je thesis,
% kan je hieronder verder.
% ----------------------- Eerste pagina's -------------------------
% Hier kan je inhoudsopgave, voorwoord en dergelijke kwijt.
% -----------------------------------------------------------------

% !TeX root = ../../thesis.tex
\chapter*{Preface}                                  \label{ch:preface}

\ldots


%%%%%%%%%%%%%%%%%%%%%%%%%%%%%%%%%%%%%%%%%%%%%%%%%%
% Keep the following \cleardoublepage at the end of this file, 
% otherwise \includeonly includes empty pages.
\cleardoublepage

% vim: tw=70 nocindent expandtab foldmethod=marker foldmarker={{{}{,}{}}}

% !TeX root = ../../thesis.tex
\chapter*{Abstract}                                 \label{ch:abstract}

\ldots


%%%%%%%%%%%%%%%%%%%%%%%%%%%%%%%%%%%%%%%%%%%%%%%%%%
% Keep the following \cleardoublepage at the end of this file, 
% otherwise \includeonly includes empty pages.
\cleardoublepage

% vim: tw=70 nocindent expandtab foldmethod=marker foldmarker={{{}{,}{}}}

% !TeX root = ../../thesis.tex
\chapter*{Beknopte samenvatting}

\ldots

%%%%%%%%%%%%%%%%%%%%%%%%%%%%%%%%%%%%%%%%%%%%%%%%%%
% Keep the following \cleardoublepage at the end of this file, 
% otherwise \includeonly includes empty pages.
\cleardoublepage

% vim: tw=70 nocindent expandtab foldmethod=marker foldmarker={{{}{,}{}}}

\tableofcontents
\listoffigures
\listoftables

\rmfamily
\setcounter{page}{0}
\pagenumbering{roman}

\newpage
% ----------------------- Eigenlijke thesis -----------------------
% Vanaf de inleiding/het eerste hoofdstuk.
% -----------------------------------------------------------------
\setcounter{page}{0}
\pagenumbering{arabic}
% !TeX root = ../../thesis.tex
\chapter{This is introduction}\label{ch:introduction}


% Illustration on how to refer to your papers when using biblatex
% (see second line in thesis.tex to activate biblatex)
%\definecolor{shadecolor}{gray}{0.85}
%\begin{shaded}
%This chapter was previously published as:\\
%\fullcite{VandenBroeck2011IJCAI}
%\newpage
%\end{shaded}

% Some dummy code to make sure bibtex does not complain.
\section{Context}

\section{Problem?}
\subsection{Dial-a-ride with deadlines?}
\subsection{Interdependent task allocation problem?}
\section{Goal?}
\subsection{Contributions?}
\subsection{Overview?}




%%%%%%%%%%%%%%%%%%%%%%%%%%%%%%%%%%%%%%%%%%%%%%%%%%
% Keep the following \cleardoublepage at the end of this file, 
% otherwise \includeonly includes empty pages.
\cleardoublepage

% vim: tw=70 nocindent expandtab foldmethod=marker foldmarker={{{}{,}{}}}

% !TeX root = ../../thesis.tex
\chapter{Literature study}

\section{Pickup and delivery problem}
\subsection{General PDP}
\subsection{Dial-a-ride with deadlines} \label{DAR-WD}

\section{MultiAgent Systems}

\section{Anticipatory systems (Swarm intelligence)}
\subsection{Belief - desire - intention systems}

\section{Delegate MAS}
\subsection{DMAS Coordination}
\subsection{DMAS Patterns}

\section{Task Allocation}
\subsection{Task Allocation state of the art}
\subsection{Task Allocation using DMAS}

\section{Interdependent task allocation}
\subsection{MultiRobot single task allocation}
\subsection{MultiRobot multiple tasks allocation}
\subsection{State of the art}


\section{RinSim Simulator}
RinSim is een simulator ontwikkeld in de Imec-Distrinet onderzoeksgroep van KuLeuven, door van Lon en Holvoet \cite{vanLon2012saso}. RinSim is specifiek ontwikkeld voor onderzoek rond het General PDP-probleem met time windows in academische kringen. Zoals besproken in \cite{DAR-WD} kan een Dial-a-Ride probleem met deadline worden omgezet in een PDPTW probleem.\\

RinSim is gericht op het testen van logistieke problemen met behulp van Multi-agent systemen. RinSim laat toe om GPDP-problemen te configureren en verschillende oplossingsmethodes te testen en via statistieken te vergelijken. Het belangrijkste kenmerk van RinSim is het onderscheidt tussen het probleemgebied en de oplossing zoals te zien op figuur TODO \cite{Figure?}. De belangrijkste elementen uit de scenario’s worden gemodelleerd als agenten. De standaardmodellen kunnen uitgebreid worden gebaseerd op de noden van de te testen oplossingsmethodes. De volgende modellen worden gebruikt of uitgebreid voor het configureren van de real-time omgeving beschreven in \cite{CaseStudy}.
TODO Add Figure
\begin{itemize}
\item \textbf{TimeModel}\\
Het \textbf{TimeModel} vormt de basis van RinSim. Dit model laat toe om de simulator op basis van een ticklength de verschillende elementen die op dit model geabonneerd zijn een actie te laten uitvoeren. In andere woorden is de TimeModel verantwoordelijk voor de passerende tijd in de simulator.
\item \textbf{RoadModel}\\
Een \textbf{RoadModel} is een interface die verschillende voorstelling van de omgeving voorstelt in bijvoorbeeld een graaf of een vlak. In een vlak kunnen de agenten zich verplaatsen op het volledige vlak in rechte lijnen terwijl een graaf voorstelling zich meer op een traditioneel straten netwerk gedraagt. Alle RoadModels hebben ook maximale snelheden die moeten worden gerespecteerd door de \textbf{MovingRoadUser} agenten. 
\item \textbf{PDP-Model}\\
De \textbf{PDP-model} bevat de implementaties van de Pickup-and-delivery problemen met time-windows en verzorgt de ophaal- en afleveroperaties van de agenten, gekoppeld met de tijdsduur en capaciteiten van de agenten.
\item \textbf{ComModel}\\
Het \textbf{ComModel} laat communicatie toe tussen verschillende agenten die dit model implementeren. Om een realistische simulatie te maken kunnen classes die van dit model gebruik maken een maximum afstand instellen voor communicatie. Deze afstand wordt in het RoadModel afgebeeld als het aantal hops die kunnen gemaakt worden over verschillende noden.
\item \textbf{StatsTracker}\\
De \textbf{statsTracker} module is het centrale informatieverzamelingspunt van RinSim, door het uitbreiden van deze module kan de verkregen informatie van RinSim verwerkt en opgeslagen worden.
\end{itemize}

Hiernaast wordt gebruikt gemaakt van de GUI-model van RinSim om een voorstelling van het probleem te maken. Hiernaast worden ook demo’s gemaakt van de voorgestelde oplossing. De voorgestelde oplossing zoals voorgesteld in \cite{VoorgesteldeOplossing} wordt vrij ontwikkeld naast deze standaardmodellen die door RinSim wordt geleverd.



%%%%%%%%%%%%%%%%%%%%%%%%%%%%%%%%%%%%%%%%%%%%%%%%%%
% Keep the following \cleardoublepage at the end of this file, 
% otherwise \includeonly includes empty pages.
\cleardoublepage

% vim: tw=70 nocindent expandtab foldmethod=marker foldmarker={{{}{,}{}}}


% !TeX root = ../../thesis.tex
\chapter{Casestudy: haven agv case}
\section{Situation}

\section{Problems}

\section{Requirements}
\subsection{Functional requirements}
\subsection{?}


%%%%%%%%%%%%%%%%%%%%%%%%%%%%%%%%%%%%%%%%%%%%%%%%%%
% Keep the following \cleardoublepage at the end of this file, 
% otherwise \includeonly includes empty pages.
\cleardoublepage

% vim: tw=70 nocindent expandtab foldmethod=marker foldmarker={{{}{,}{}}}

% !TeX root = ../../thesis.tex
\chapter{Interdependent sequential task allocation }                                  \label{ch:Interdependent sequential task allocation}

\ldots


%%%%%%%%%%%%%%%%%%%%%%%%%%%%%%%%%%%%%%%%%%%%%%%%%%
% Keep the following \cleardoublepage at the end of this file, 
% otherwise \includeonly includes empty pages.
\cleardoublepage

% vim: tw=70 nocindent expandtab foldmethod=marker foldmarker={{{}{,}{}}}

% !TeX root = ../../thesis.tex
\chapter{Experiment/Simulatie}                                  \label{ch:experiment}

\section{Hypothesis}
\section{Evaluation criteria}
\section{Design}
\section{Result}
\section{Conclusion}


\ldots


%%%%%%%%%%%%%%%%%%%%%%%%%%%%%%%%%%%%%%%%%%%%%%%%%%
% Keep the following \cleardoublepage at the end of this file, 
% otherwise \includeonly includes empty pages.
\cleardoublepage

% vim: tw=70 nocindent expandtab foldmethod=marker foldmarker={{{}{,}{}}}


% !TeX root = ../../thesis.tex
\chapter{Related work}                                  \label{ch:Related work}

\ldots


%%%%%%%%%%%%%%%%%%%%%%%%%%%%%%%%%%%%%%%%%%%%%%%%%%
% Keep the following \cleardoublepage at the end of this file, 
% otherwise \includeonly includes empty pages.
\cleardoublepage

% vim: tw=70 nocindent expandtab foldmethod=marker foldmarker={{{}{,}{}}}

% !TeX root = ../../thesis.tex
\chapter{Conclusion}\label{ch:conclusion}

\ldots


%%%%%%%%%%%%%%%%%%%%%%%%%%%%%%%%%%%%%%%%%%%%%%%%%%
% Keep the following \cleardoublepage at the end of this file, 
% otherwise \includeonly includes empty pages.
\cleardoublepage

% vim: tw=70 nocindent expandtab foldmethod=marker foldmarker={{{}{,}{}}}

% BibTex
\newpage
\bibliography{allpapers}
\bibliographystyle{acm}


\newpage
% ----------------------- Achterblad ------------------------------
% Vergeet niet de tekst aan te passen:
% - Afdeling
% - Adres van de afdeling
% - Telefoon en faxnummer
% -----------------------------------------------------------------
\thispagestyle{empty}
\sffamily
%
\begin{textblock}{191}(113,-11)
{\color{blueline}\rule{160pt}{5.5pt}}
\end{textblock}
%
\begin{textblock}{191}(168,-11)
{\color{blueline}\rule{5.5pt}{59pt}}
\end{textblock}
%
\begin{textblock}{183}(-24,-11)
\textblockcolour{}
\flushright
\fontsize{7}{7.5}\selectfont
\textbf{AFDELING}\\
Straat nr bus 0000\\
3000 LEUVEN, BELGI\"{E}\\
tel. + 32 16 00 00 00\\
fax + 32 16 00 00 00\\
www.kuleuven.be\\
\end{textblock}
%
\begin{textblock}{191}(154,-7)
\textblockcolour{}
\includegraphics*[height=16.5truemm]{sedes}
\end{textblock}
%
\begin{textblock}{191}(-20,235)
{\color{bluetitle}\rule{544pt}{55pt}}
\end{textblock}
\end{document}
